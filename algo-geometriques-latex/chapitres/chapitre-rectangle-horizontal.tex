%!TEX encoding = UTF-8 Unicode
%!TEX root = ../algos-geometriques.tex


\chapterLabel{Rectangle horizontal}{rectangleHorizontal}

Par rectangle « horizontal », on entend un rectangle dont les côtés sont parallèles aux axes. Un tel rectangle est décrit par le type \texttt{CGRect}.

Dans ce chapitre, des additions à ce type sont présentées. Elles sont définies comme des extensions du type \texttt{CGRect}, et implémentées dans \texttt{extension-CGRect.swift}.




\section{Construction d'un rectangle à partir de deux points}

Cet initialiseur permet de construire un rectangle à partir de deux points ; si les points sont confondus, la taille du rectangle est nulle.


\begin{lstlisting}
extension CGRect {
  init (point p1: CGPoint, point p2: CGPoint) {
    origin = CGPoint (x: min (p1.x, p2.x), y: min (p1.y, p2.y))
    size = CGSize (width: abs (p1.x - p2.x), height: abs (p1.y - p2.y))
  }
}
\end{lstlisting}

